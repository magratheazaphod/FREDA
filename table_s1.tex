\documentclass[singlecolumn,11pt]{pnas-new}
% Use the lineno option to display guide line numbers if required.
% Note that the use of elements such as single-column equations
% may affect the guide line number alignment. 

%Many authors find it useful to organize their manuscripts with the following order of sections;  Title, Author Affiliation, Keywords, Abstract, Significance Statement, Results, Discussion, Materials and methods, Acknowledgments, and References. Other orders and headings are permitted.

\usepackage[super]{nth} %adds superscripts for things like 20th

\templatetype{pnasresearcharticle} % Choose template 
% {pnasresearcharticle} = Template for a two-column research article
% {pnasmathematics} = Template for a one-column mathematics article
% {pnasinvited} = Template for a PNAS invited submission

% Please give the surname of the lead author for the running footer
\leadauthor{Day} 

\begin{document}

\dates{This manuscript was compiled on \today}

\makeatletter 
\renewcommand{\thetable}{S\@arabic\c@table}
\makeatother
\setcounter{table}{0}

%% SUPPLEMENTARY TABLE S1 - statistical significance of changes in distribution, 1980-2007 v 1951-1979
\begin{table}[p]

\centering

\caption{Statistical significance (expressed as $p$-value) of change in distribution of latitude and intensity of rainbands between 1951-1979 and 1980-2007, as calculated by a bootstrap Kolmogorov-Smirnov (K-S) and bootstrap Anderson-Darling (A-D) test, each with 10,000 iterations. Statistically significant changes at the 95\%/99\% level are indicated by bold font/bold font and asterisk respectively.}

\begin{tabular}{ l c c c c}
												& \multicolumn{2}{c}{Latitude ($^\circ$)} & \multicolumn{2}{c}{Intensity  (mm day$^{-1}$)} \\
	 \textbf{Time Period} 							& K-S 			& A-D 			& K-S 	& A-D \\
	 \hline
	\textbf{Spring Rains} (Mar 1-Apr 30, 60-120)  		& .086			& .037			& .19	& .083 \\
	\textbf{Pre-Meiyu} (May 1-Jun 9, 121-160)  		& .24 			&  .086 			& .94	& .90 \\
	\textbf{Meiyu} (Jun 10-Jul 19, 161-120)			& .30			&  .21			&  .25	& .40 \\	
	\textbf{Post-Meiyu} (Jul 20-Sep 30) 				& \textbf{.0073}	&  \textbf{.0018*}  	&  .28 	& .24 \\
	\textbf{Post-Meiyu} (Jul 20-Sep 30), $>28^{\circ}N$   & \textbf{.0010*}	&  \textbf{.0001*} 	&  .04 	& .10 \\	
	\textbf{Post-Meiyu} (Jul 20-Sep 30), $<28^{\circ}N$   & .38			&  .33			&  .53	& .62 \\	
	\textbf{Fall Rains} (Oct 1-Nov 16) 					& .23 			&  .15			&  .94 	& .83 \\	
	\textbf{Full Year} (1-365)						& .075			&  \textbf{.016} 	&  .26 	& .12 \\	
	
\end{tabular}
\end{table}

\end{document}